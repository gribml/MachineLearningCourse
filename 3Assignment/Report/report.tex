\documentclass[a4paper,12pt,oneside,final]{report}
\usepackage[pdftex]{graphicx}
\usepackage{amssymb}
\usepackage{epstopdf}
\usepackage[utf8]{inputenc}
\usepackage{titlesec}
\usepackage[titletoc]{appendix}
\titleformat{\chapter}[hang]{\bf\Huge}{\thechapter}{1cm}{}

\usepackage[colorlinks=true]{hyperref}
\hypersetup{urlcolor=blue,linkcolor=black,citecolor=black,colorlinks=true}
\bibliographystyle{plain}

\pagestyle{plain}
% -------------------- this stuff for code --------------------

\usepackage{anysize}
\marginsize{30mm}{30mm}{20mm}{20mm}

\newenvironment{formal}{%
  \def\FrameCommand{%
    \hspace{1pt}%
    {\color{blue}\vrule width 2pt}%
    {\color{formalshade}\vrule width 4pt}%
    \colorbox{formalshade}%
  }%
  \MakeFramed{\advance\hsize-\width\FrameRestore}%
  \noindent\hspace{-4.55pt}% disable indenting first paragraph
  \begin{adjustwidth}{}{7pt}%
  \vspace{2pt}\vspace{2pt}%
}
{%
  \vspace{2pt}\end{adjustwidth}\endMakeFramed%
}

\newenvironment{changemargin}[2]{\begin{list}{}{%
\setlength{\topsep}{0pt}%
\setlength{\leftmargin}{0pt}%
\setlength{\rightmargin}{0pt}%
\setlength{\listparindent}{\parindent}%
\setlength{\itemindent}{\parindent}%
\setlength{\parsep}{0pt plus 1pt}%
\addtolength{\leftmargin}{#1}%
\addtolength{\rightmargin}{#2}%
}\item }{\end{list}}

\usepackage{color}
\usepackage{dsfont}
\usepackage[bitstream-charter]{mathdesign}
\usepackage[scaled]{helvet}
\usepackage{inconsolata}


\definecolor{colKeys}{rgb}{0,0,0.9} 
\definecolor{colIdentifier}{rgb}{0,0,0} 
\definecolor{colString}{rgb}{0.7,0,0} 
\definecolor{colComments}{rgb}{0,0.6,0} 
\usepackage{listings}
\lstset{
  stringstyle=\color{colString},
  keywordstyle=\color{colKeys},
  identifierstyle=\color{colIdentifier},
  commentstyle=\color{colComments},
  numbers=left,
  tabsize=4,
  frame=single,
  breaklines=true,
  basicstyle=\small\ttfamily,
  numberstyle=\tiny\ttfamily,
  framexleftmargin=0mm,
  xleftmargin=7mm,
  xrightmargin=7mm,
  frameround={tttt},
  captionpos=b
}

\usepackage{mathtools}
\usepackage{amsthm}
\newtheorem{definition}{Definition}
\newtheorem{theorem}{Theorem}
\DeclareMathOperator*{\argmin}{ArgMin\ }
\DeclareMathOperator*{\argmax}{ArgMax\ }

\usepackage{algorithm}
\usepackage{algorithmic}

\usepackage[usenames,dvipsnames]{xcolor}
\makeatletter
\DeclareRobustCommand{\em}{%
  \@nomath\em \if b\expandafter\@car\f@series\@nil
  \normalfont \else \color{BrickRed} \bfseries \fi}
\makeatother

%% Headers and footers
\usepackage{fancyhdr}
\usepackage[section]{placeins}
\pagestyle{fancy}
\fancyhf{}
\addtolength{\headwidth}{30pt}
\addtolength{\headwidth}{30pt}
\renewcommand{\headrulewidth}{0.4pt} % thickness of the header line
\renewcommand{\footrulewidth}{0.4pt} % thickness of the footer line
\renewcommand{\chaptermark}[1]{\markboth{#1}{#1}} % chapter name
\renewcommand{\sectionmark}[1]{\markright{\thesection\ #1}}  % section name
\lhead[\fancyplain{}{\bf\thepage}]{\fancyplain{}{\bf\rightmark}} % display header
\rhead[\fancyplain{}{\bf\leftmark}]{\fancyplain{}{}} % display header
\fancyfoot[C]{\bf\thepage} % display footer (page number)
\fancyfoot[R]{\bf\today} % display footer (date)
\fancypagestyle{plain}{ 
	\fancyhead{} \renewcommand{\headrulewidth}{0pt}
}
\newcommand{\clearemptydoublepage}{\newpage{\pagestyle{plain}\cleardoublepage}}

\usepackage[T1]{fontenc}
\usepackage{enumerate}
\usepackage{afterpage,lastpage,fancyhdr}
\usepackage[includeheadfoot,margin=2.5cm]{geometry}
\geometry{letterpaper}                   % ... or a4paper or a5paper or ... 

\DeclareGraphicsRule{.tif}{png}{.png}{`convert #1 `dirname #1`/`basename #1 .tif`.png}

\makeatletter \def\thickhrulefill{\leavevmode \leaders \hrule height 1pt\hfill
\kern \z@} \renewcommand{\maketitle}{
    \begin{titlepage}
    \let\footnotesize\small \let\footnoterule\relax \parindent \z@ \reset@font
    \null\vfil
    \vspace{-20mm}
    \begin{center}
    {\small \scshape Imperial College London \\ Department of Computing}
    \end{center}
    \vspace{0.5cm}
	\begin{minipage}{\textwidth}
		\vspace{1cm}
		\noindent\rule[0ex]{\textwidth}{4pt} \\
		\flushright
		\center
		\@title
		\\ \vspace{4mm}
		\noindent\rule[0ex]{\textwidth}{4pt} \\
	\end{minipage}
	\vspace{1.5cm}
	\begin{minipage}{\textwidth}
		\flushright
		{\bfseries}
		\vspace{7mm}
		\flushleft
		\@author.\\
	\end{minipage}
	\vspace{0.5cm}
	\begin{center}
		\includegraphics[width=70mm,]{pictures/logo_imperial_college_london.png}
	\end{center}
	\vspace{\stretch{1}}
	\vspace{50mm}
		\flushleft
		{\bfseries}
		Module leader \& Lecturer: Dr Maja \textsc{Pantic}. \\
		{\small \scshape \@date }.
		\vspace{0.1cm}
		\rule{\linewidth}{.5pt}
  \end{titlepage}
  \setcounter{footnote}{1}
  \setcounter{page}{2}
}


\author{
    Sedef Ozlen (so512, s5), \\ 
    Paul Gribelyuk (pg1312, a5), \\
    Jean Kossaifi (jk712, a5), \\ 
    Romain Brault (rb812, a5)
}
\makeatother
\title{\Huge Machine Learning \\ Artificial Neural Network \\ Coursework 3}
\date{\today}


\usepackage{amsmath}
\begin{document}
\maketitle
\tableofcontents
\listoffigures

\chapter{Introduction}
\paragraph{}
The goal of this coursework is to understand how to use the MATLAB Neural Network Toolbox and the variety of methods which can be used to classify the same dataset used when implementing decision trees.  We implemented both a single 6-way classifier as well as 6 different binary classifiers, where the binary classifiers were later combined to produce a single classification of the data.
\paragraph{}
The MATLAB Neural Network Toolbox provides a great deal of flexibility in specifying the exact network topology, performance measurement function, learning rate (when computing weights via stochastic gradient descent, transfer/activation function, and training function, and we learned the relative merits of these tools throughout the course of the assignment.  The most important lesson we learned is that this machine learning technique is extremely powerful, but the search for an optimal network is extremely computationally intensive.  The advent of back propagation in conjunction with stochastic gradient descent in the 1970s, when optimizing node weights, accelerated the study of neural networks for real-world applications, spawning the use of a variety of \emph{performance measures}, \emph{activation functions} (a.k.a. transfer functions), and \emph{training functions}.  In this report, we will show our results for the classification rate of the sample data for different network topologies, activation and transfer functions, as well as the effects of a variable learning rate. 

\chapter{Naïve networks}
\paragraph{}
First, we built two neural networks classifier to find a predictive model for emotions from AU.
\section{Creating two types of neural network}
\paragraph{} The function:
\begin{changemargin}{-5mm}{-5mm}
\begin{lstlisting}[language=Matlab, frame=single]
function [ totalnet, binaryNets ] = partFour(x, y)
\end{lstlisting}
\end{changemargin}
located in \textit{partFour.m} takes the database as imput argument and return the single six-output neural network (\textit{totalnet}), and a six single-output neural networks (\textit{binaryNets}). It uses an internal function:
\begin{changemargin}{-5mm}{-5mm}
\begin{lstlisting}[language=Matlab, frame=single]
function [ network ] = buildNetwork( HiddenLayer, epochs, dataSplit, x, y, perf,  lr, tf, trainFn)
\end{lstlisting}
\end{changemargin}
Which build a single network according to argument parameters. 
\begin{itemize}
\item {\bf \textit{HiddentLayer} } is a vector controling the number of hidden layers of the neural network, as specified by the function \textit{feedforwardnet}. 
\item {\bf \textit{Datasplit} } is a vector expressing the fraction of the database which must be used as training set, validation set and testing set. The sum of the elements of the vector must be equal to $1$.
\item {\bf \textit{epoch} } controls the number of iterations used by the training algorithm ( )
\end{itemize}
\paragraph{}
\section{Testing the networks}

\chapter{Optimised Networks}
\section{Optimizing the topology and parameters of the networks}
\section{Train a network}
\section{Performing 10 fold cross-validation}
\section{Plotting performance measure}

\chapter{Conclusion}

\bibliographystyle{alpha}
\bibliography{biblio.bib}

\begin{appendices}

\end{appendices}

\end{document}  
